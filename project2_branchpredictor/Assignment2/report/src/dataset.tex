
\begin{section}{Project Description}
This project simulates and analyses bimodal and gshare branch predictors on various benchmark programs.
\end{section}


\begin{section}{Results}

    \begin{subsection}{Bimodal Predictor}

        \begin{center}
        (a)
        \end{center}

    The adjoining plot depicts the effect of number of bits, ($m$), used for prediction, on the misprediction rate for two different traces. 
    
   
        \begin{figure}[h] % Positioning option: h (here)
            \centering
            \begin{subfigure}[b]{0.45\textwidth}
                \centering
                \includegraphics[width=\textwidth]{figures/fig1/fig1a1.png} % Replace with your first image
                \label{fig:pic1}
            \end{subfigure}
            \hfill
            \begin{subfigure}[b]{0.45\textwidth}
                \centering
                \includegraphics[width=\textwidth]{figures/fig1/fig1a2.png} % Replace with your second image
                \label{fig:pic2}
            \end{subfigure}
            \caption{Miss prediction rate for varying number of prediction bits $m$}
            \label{fig:twopictures}
        \end{figure}

        Some general observations and similarities between the plots is explained below:
        \begin{itemize}
            \item Clearly the misprediction rate decreases as we increase the number of bits $m$ used for prediction. With lesser $m$, there are collisions that happen among different branches having identical last $m+2$ bits. This mixes and garbles up the counters, leading to higher mispredictions. Having more number of bits decreases the number of collisions and thus, the misprediction rate.
            \item Clearly, there is a case of diminishing returns for both the graphs. In the left figure, we see that initially, increasing $m$ by 2 decreases the miss prediction rate by 4\%. However, at higher $m$, the same increase causes a decrease of less than 2\% for miss prediction rate. Similarly, miss prediction decrease falls from 0.1\% initially to about 0.02\% at higher $m$.   % Maybe you can add the intuitive explanation for this here
        \end{itemize}

        Some differences between the two plots are:
        \begin{itemize}
            \item Firstly, it is noteworthy that the \texttt{gcc} trace has misprediction rate in the range of 10-30\%, while the \texttt{jpeg} trace, has only 7-8\% mispredictions. This means that the \texttt{gcc} trace inherently has more \textit{unpredictable} branches compared to \texttt{jpeg} for the same number of prediction bits used for bimodal predictor. 
            \item Secondly, the fractional decrease in the misprediction rate, by increasing $m$, is much higher in \texttt{gcc} (around ~15\%) than in \texttt{jpeg} (~5-6\%). This proves that despite having more unpredictable branches than \texttt{jpeg}, the improvement in misprediction rate with increasing $m$ is much more for \texttt{gcc}. 
        \end{itemize}

        \begin{center}
            (b) 
        \end{center}
        Given $m$ number of bits used for indexing into the predictor table, number of storage bits required for the predictor table is $2^{m+2}$. This is because each entry has to be a 2-bit counter. Thus, with 16KB budget, one can have $m$ as atmost $16$. But from the graphs we see that the improvement from $m=11$ to $m=12$ is not appreciable. However in the case of \texttt{jpeg} trace the increase in accuracy from 10 to 11 is sizeable. Further, with $m=11$ only 1KB of memory is used. Thus, $\mathbf{m=11}$ is a reasonable choice balancing misprediction rate, storage and power.
    \end{subsection}

    \begin{subsection}{Gshare Predictor}
        
        \begin{center}
            (a)
        \end{center}

        The figure below shows the misprediction rates of the Gshare branch predictor for different values of $m$ (lookup bits from the program counter address) and $n$ (global branch history register).

        \begin{figure}[h] % Positioning option: h (here)
            \centering
            \begin{subfigure}[b]{0.45\textwidth}
                \centering
                \includegraphics[width=\textwidth]{figures/fig2/gcc_misprediction_plot.png} % Replace with your first image
                \label{fig:pic1}
            \end{subfigure}
            \hfill
            \begin{subfigure}[b]{0.45\textwidth}
                \centering
                \includegraphics[width=\textwidth]{figures/fig2/jpeg_misprediction_plot.png} % Replace with your second image
                \label{fig:pic2}
            \end{subfigure}
            \caption{Miss prediction rate for varying number of prediction bits $m$ and global branch register size $n$}
            \label{fig:twopictures}
        \end{figure}

        Some general observations, similarities and differences from the above plots are described below:
        \begin{itemize}
            \item For a given size $n$ of the global branch history register, we see that the misprediction rate decreases if we increase the number of prediction bits $m$ considered from the address. As explained in the previous plot, this is because of the reduced number of address collisions in the branch prediction buffer.
            \item For a given $m$, it is expected that increasing the global history register size $m$, decreases the misprediction rate by better utilising the global branch history. While this trend is clearly false in the case of \texttt{gcc} trace, it somewhat holds for \texttt{jpeg} trace. This is probably because, the initial setup of the global branch history takes time for larger values of $n$. Thus, there may be a lot of initial mispredictions. This effect would probably decrease with a larger trace.
        \end{itemize}

        \begin{center}
            (b)
        \end{center}
        One important observation here is that, for Gshare predictor, increasing $n$ does not increase the storage required for the predictor table. Further, it has negligible increase on the power also. Thus, a larger value of $n$ is preferred. From the figure we observe that $m=11$ gives a reasonable misprediction rate. Any lesser value of $m$ would increase the mispredictions by atleast 35\% and an increase in $m$ for \texttt{jpeg} is observed to, unintuitively, increase the mispredictions. For this $m$, considering both the plots we observe that $n=4$ is a good choice as it has second best misprediction rate for both the traces. Thus, $\mathbf{m=11}$ and $\mathbf{n=4}$, is the preferred choice. This choice has reasonable storage requirement of $1KB$ and achieves a misprediction rate of about 14\% and 7.25\% for the \texttt{gcc} and \texttt{jpeg} traces respectively.

    \end{subsection}

\end{section}