\usepackage[utf8]{inputenc} % Kodierung
\usepackage[english]{babel} % Sprache
\usepackage{graphicx} % immer benötigt für das Einbinden von Graphiken
\usepackage{parskip} % Für den Abstand zwischen 2 Absätzen.
\setlength{\parskip}{12pt plus80pt minus10pt} % Genaue Einstellung von parskip
\usepackage{csquotes} % Für ordentlichen Anführungszeichen
\usepackage[citestyle=numeric-comp,
sorting=none]{biblatex} % bibtex backend für Literaturverzeichnis
\usepackage{xcolor} %Paket für farbige Texte
\usepackage{setspace} %Paket fuer groessere Abstaende, zB fuer das Titelblatt
\addbibresource{literatur/bibliography.bib} % Einbinden der Literatur
\usepackage{url}
\usepackage[activate={true,nocompatibility},
	final,
	tracking=true,
	kerning=true,
	expansion=true,
	spacing=true,
	factor=1050,
	stretch=25,
	shrink=10]{microtype} % Für die Feineinstellung der Zeichensetzung.
\usepackage{booktabs}
\usepackage{fancyvrb}
\usepackage[hidelinks]{hyperref} % Klickbare aber nicht markierte Links im PDF
\usepackage{fancyhdr} % Für schönere Kopf-/Fußzeilen und Fußnoten.
\usepackage[right=2.5 cm, left=2.5 cm, top=2.5 cm, bottom=2.5 cm]{geometry} % Seitenränder
\usepackage[allow-number-unit-breaks=true]{siunitx}

%% ADDED BY SNEHADEEP %%
\usepackage{url}
\usepackage{graphicx}
\usepackage{subcaption}
\usepackage{siunitx}
% \sisetup{output-exponent-marker=\ensuremath{\mathrm{e}}}
\usepackage{amsmath}
\usepackage{placeins}
\usepackage{amssymb}
\usepackage{algpseudocode}
\usepackage{algorithm}

\newcommand\norm[1]{\lVert#1\rVert}
\DeclareMathOperator{\sign}{sign}